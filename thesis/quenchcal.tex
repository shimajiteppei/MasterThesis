\chapter{局所分離クエンチで用いた数値計算}
\ref{chap:singlquench},\ref{chap:doublequench}章でのEEの数値計算に用いたコードを載せておく。(修士論文発表会後にhttps://github.com/tomoesaturn/MasterThesis で公開する予定である。)

境界共形場理論での相関関数の計算には二重化のトリックを用いていたが、これを数値的に実現するためには複素関数のbranchに注意してその定義域を適切にとる必要がある。最終的に我々は$x+i\tau\to x-t$の解析接続をしたいので、数値計算では最初から$w=x-t,\overline{w}=x+t$を用いて、$t=0$で$w=\overline{w}$となるようにbranchの取り方を決める。

以下のコードではPythonのmpmathライブラリを用いて計算している。mpmathライブラリでのテータ関数の定義は、我々の定義と$\nu$が$\pi$倍だけずれていることに注意する。これによってテータ関数の$\nu$微分も$\pi$倍ずれる。

\begin{lstlisting}
import mpmath as mp
from mpmath import j,pi,exp,jtheta,re,im,sin,sinh,log,atan,tanh,cos,diff,qp


"""
SINGLE SPLITING QUENCH
"""
# conformal map for single splitting quench. zeta=exp(i*theta)
def theta(x,t,a):
	if x>0:
		return atan(-(x-t)/a)
	else:
		return -pi+atan(-(x-t)/a)

def thetaB(x,t,a):
	if x>0:
		return atan(-(x+t)/a)
	else:
		return -pi+atan(-(x+t)/a)

#EE for dirac fermion
def S_single_dirac(x1,x2,t,a):
	t1,t2,tb1,tb2 = theta(x1,t,a), theta(x2,t,a), thetaB(x1,t,a), thetaB(x2,t,a)
	z1,z2,zb1,zb2 = j*exp(j*t1), j*exp(j*t2), j*exp(-j*tb1), j*exp(-j*tb2)

	deriv = a**2 / ( exp(j*(t1-tb1+t2-tb2)/2) * cos(t1)*cos(t2)*cos(tb1)*cos(tb2) )
	conn = (z1-z2)*(zb1-zb2)
	disconn = (z1-zb1)*(z2-zb2)
	cross = (z1-zb2)*(z2-zb1)

	return re(log( deriv * conn * disconn / cross ))/6 - log((x2-x1)**2)/6

#EE for holographic CFT
def S_single_hol_conn(x1,x2,t,a):
	t1,t2,tb1,tb2 = theta(x1,t,a), theta(x2,t,a), thetaB(x1,t,a), thetaB(x2,t,a)
	z1,z2,zb1,zb2 = j*exp(j*t1), j*exp(j*t2), j*exp(-j*tb1), j*exp(-j*tb2)

	deriv = a**2 / ( exp(j*(t1-tb1+t2-tb2)/2) * cos(t1)*cos(t2)*cos(tb1)*cos(tb2) )
	conn = (z1-z2)*(zb1-zb2)

	return re(log( deriv * conn ))/6 - log((x2-x1)**2)/6

def S_single_hol_disconn(x1,x2,t,a):
	t1,t2,tb1,tb2 = theta(x1,t,a), theta(x2,t,a), thetaB(x1,t,a), thetaB(x2,t,a)
	z1,z2,zb1,zb2 = j*exp(j*t1), j*exp(j*t2), j*exp(-j*tb1), j*exp(-j*tb2)

	deriv = a**2 / ( exp(j*(t1-tb1+t2-tb2)/2) * cos(t1)*cos(t2)*cos(tb1)*cos(tb2) )
	disconn = (z1-zb1)*(z2-zb2)

	return re(log( deriv * disconn ))/6 - log((x2-x1)**2)/6

def S_single_hol(x1,x2,t,a):
	t1,t2,tb1,tb2 = theta(x1,t,a), theta(x2,t,a), thetaB(x1,t,a), thetaB(x2,t,a)
	z1,z2,zb1,zb2 = j*exp(j*t1), j*exp(j*t2), j*exp(-j*tb1), j*exp(-j*tb2)

	deriv = a**2 / ( exp(j*(t1-tb1+t2-tb2)/2) * cos(t1)*cos(t2)*cos(tb1)*cos(tb2) )
	conn = (z1-z2)*(zb1-zb2)
	disconn = (z1-zb1)*(z2-zb2)

	return min([re(log( deriv * conn )),re(log( deriv * disconn ))])/6 - log((x2-x1)**2)/6


"""
DOUBLE SPLITING QUENCH
"""
# theta/eta function
def jt1(nu,tau):
	return jtheta(1,pi*nu,exp(pi*j*tau),0)

def jt1d(nu,tau):
	return jtheta(1,pi*nu,exp(pi*j*tau),1)

def eta(tau):
	return exp(j*pi*tau/12)*qp(exp(2*pi*j*tau),exp(2*pi*j*tau))

# conformal map from cylinder to plane, s=beta^{-1}
def w(v,s,b):
	return -j*b*( jt1d(v,j*s)/jt1(v,j*s) + jt1d(v+j*s/2,j*s)/jt1(v+j*s/2,j*s) + j )

def Dw(v,s,b):
	return diff(lambda x: w(x,s,b), v, 1)

def cutoff(s,b):
	return im(w(mp.findroot(lambda y: diff(lambda x: im(w(x-j*s/4,s,b)) ,y,1),1/2)-j*s/4,s,b))

# inverse map
def v(x,t,s,b):
	# c is atrificial parameter for findroot.
	c = 10**(-10)
	if x>b:
		return j*mp.findroot(lambda y: re(w(j*y,s,b)-(x-t)), [-s/2+c,0-c],"bisect")
	elif x<-b:
		return j*mp.findroot(lambda y: re(w(j*y,s,b)-(x-t)), [0+c,s/2-c],"bisect")
	else:
		tolerant = 10**(-20)
		normparam = 10**11
		return 1/2+j*mp.findroot(lambda y: tanh(re(w(1/2+j*y,s,b)-(x-t))/normparam), 0,tol=tolerant)

def vb(x,t,s,b):
	if abs(x)>b:
		return -v(x,-t,s,b)
	else:
		return 1-v(x,-t,s,b)

#EE for dirac fermion
def S_double_dirac(x1,x2,t,s,b):
	v1,v2,vb1,vb2 = v(x1,t,s,b),v(x2,t,s,b),vb(x1,t,s,b),vb(x2,t,s,b)

	deriv = ((2*pi)**(-4)) * Dw(v1,s,b)*(-Dw(-vb1,s,b))*Dw(v2,s,b)*(-Dw(-vb2,s,b))
	conn = jt1(v1-v2,j*s) * jt1(vb2-vb1,j*s) / eta(j*s)**6
	disconn = jt1(v1-vb1+j*s/2,j*s) * jt1(v2-vb2+j*s/2,j*s)
	cross = jt1(v1-vb2+j*s/2,j*s) * jt1(v2-vb1+j*s/2,j*s)

	return re(log( deriv * (conn * disconn / cross)**2 ))/12-log((x2-x1)**2)/6

#EE for holographic CFT
def S_double_hol_conn(x1,x2,t,s,b):
	v1,v2,vb1,vb2 = v(x1,t,s,b),v(x2,t,s,b),vb(x1,t,s,b),vb(x2,t,s,b)

	deriv = (pi**(-4)) * Dw(v1,s,b)*(-Dw(-vb1,s,b))*Dw(v2,s,b)*(-Dw(-vb2,s,b))
	if s<1:
		conn = (s**2) * sinh(pi * (v1-v2)/s) * sinh(pi * (vb1-vb2)/s)
	else:
		conn =  sin(pi * (v1-v2)) * sin(pi * (vb1-vb2))

	return re(log( deriv * conn**2 ))/12-log((x2-x1)**2)/6

def S_double_hol_disconn(x1,x2,t,s,b):
	v1,v2,vb1,vb2 = v(x1,t,s,b),v(x2,t,s,b),vb(x1,t,s,b),vb(x2,t,s,b)
	#difference from mirror twist op
	p1,n1,p2,n2 = v1-vb1+j*s/2, v1-vb1-j*s/2, v2-vb2+j*s/2, v2-vb2-j*s/2

	deriv = (pi**(-4)) * Dw(v1,s,b)*(-Dw(-vb1,s,b))*Dw(v2,s,b)*(-Dw(-vb2,s,b))
	if s<1:
		g1 = min([re(sinh(pi * p1/s)**2),re(sinh(pi * n1/s)**2)])
		g2 = min([re(sinh(pi * p2/s)**2),re(sinh(pi * n2/s)**2)])
		disconnSquare = (s**4) * g1 * g2
	else:
		m1 = sin(pi * p1) * sin(pi * p2)
		m2 = sin(pi * p1) * sin(pi * n2) * exp(pi*s)
		m3 = sin(pi * n1) * sin(pi * p2) * exp(pi*s)
		m4 = sin(pi * n1) * sin(pi * n2)
		disconnSquare =  min([re(m1**2),re(m2**2),re(m3**2),re(m4**2)])

	return re(log( deriv * disconnSquare ))/12-log((x2-x1)**2)/6

def S_double_hol(x1,x2,t,s,b):
	v1,v2,vb1,vb2 = v(x1,t,s,b),v(x2,t,s,b),vb(x1,t,s,b),vb(x2,t,s,b)
	p1,n1,p2,n2 = v1-vb1+j*s/2, v1-vb1-j*s/2, v2-vb2+j*s/2, v2-vb2-j*s/2

	deriv = (pi**(-4)) * Dw(v1,s,b)*(-Dw(-vb1,s,b))*Dw(v2,s,b)*(-Dw(-vb2,s,b))
	if s<1:
		conn = (s**2) * sinh(pi * (v1-v2)/s) * sinh(pi * (vb1-vb2)/s)
		g1 = min([re(sinh(pi * p1/s)**2),re(sinh(pi * n1/s)**2)])
		g2 = min([re(sinh(pi * p2/s)**2),re(sinh(pi * n2/s)**2)])
		disconnSquare = (s**4) * g1 * g2
	else:
		conn =  sin(pi * (v1-v2)) * sin(pi * (vb1-vb2))
		m1 = sin(pi * p1) * sin(pi * p2)
		m2 = sin(pi * p1) * sin(pi * n2) * exp(pi*s)
		m3 = sin(pi * n1) * sin(pi * p2) * exp(pi*s)
		m4 = sin(pi * n1) * sin(pi * n2)
		disconnSquare =  min([re(m1**2),re(m2**2),re(m3**2),re(m4**2)])

	return min([re(log( deriv * re(conn)**2 )),re(log( deriv * disconnSquare ))])/12-log((x2-x1)**2)/6
	
"""
PLOTTING
"""
if __name__ == '__main__':

	mp.dps = 500  #dps for precision
	mp.pretty = True

	#plot of DOUBLE SPLITTING QUENCH
	b = 50
	maxtime = 300
	plotpoints = 300
	s = 0.945
	x1, x2 = 100, 200
	mp.plot([lambda t: S_double_dirac(x1, x2, t, s, b)],[0,maxtime],points=plotpoints)
	mp.plot([lambda t: S_double_hol_conn(x1, x2, t, s, b),lambda t: S_double_hol_disconn(x1, x2, t, s, b)],[0,maxtime],points=plotpoints)
	
	"""
	plot of SINGLE SPLITTING QUENCH
	"""
	s,b = 5.28,50
	a = cutoff(s,b)
	maxtime = 200
	plotpoints = 1000
	x1, x2 = 50, 150
	mp.plot(lambda t: S_single_dirac(x1, x2, t, a),[0,maxtime],points=plotpoints)
	mp.plot([lambda t: S_single_hol_conn(x1, x2, t, a),lambda t: S_single_hol_disconn(x1, x2, t, a)],[0,maxtime],points=plotpoints)
\end{lstlisting}