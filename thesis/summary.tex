\chapter{まとめと今後の展望}
\section*{まとめ}
本論文では、AdS/CFT対応の手法を利用した分離クエンチにおけるエンタングルメントのダイナミクスの解析を述べた。

\ref{chap:adscftreview}章ではAdS$_3$/CFT$_2$対応を概説した。AdS$_3$/CFT$_2$対応では、2次元共形場理論のVirasoro対称性が3次元漸近的AdS時空の境界にあるVirasoro対称性に対応し、さらに共形場理論の高温/低温状態はブラックホールが存在する/しない時空に対応する。そして共形場理論の高温状態のエントロピーはCardy公式を通してBekenstein-Hawkingエントロピーに対応する。2次元共形場理論が古典的なAdS$_3$重力理論に双対になるためには、古典極限$R_A\gg G$に対応して中心電荷が$c\gg 1$となり、また、重力理論のスペクトルに対応して共形場理論のスペクトルも``sparse"になる必要がある。また、例えば共形境界条件を課した上半平面上や有限長さの円筒の境界共形場理論は、二重化のトリックを用いることで半分のVirasoro対称性をもった平面上の共形場理論に対応する。この境界共形場理論の重力双対は、AdS時空を``半分"に分けたものに自然に対応すると考えるのがAdS/BCFTの処方である。

\ref{chap:EEreview}章では2次元共形場理論のエンタングルメントエントロピーを概説した。場の理論のエンタングルメントエントロピーとは空間領域を2つに分けたときに定義され、真空ですら紫外発散するエンタングルメントエントロピーを持つ。2次元共形場理論の真空の連結区間に対するエンタングルメントエントロピーは、その区間の長さと中心電荷のみで決まる。とくに重力双対をもつような共形場理論に対しては、共形場理論のエンタングルメントエントロピーはAdS空間の極小測地線の長さに対応するというのが笠-高柳公式の主張である。笠-高柳公式の強力なところは、長さと中心電荷だけでは決まらない非連結区間のエンタングルメントエントロピーを測地線の長さの計算に対応づけたり、混合状態のエンタングルメントエントロピーをBekenstein-Hawkingエントロピーに関係づけたりすることができる点である。

\ref{chap:singlquench}章では我々の論文\cite{Shimaji:2018czt}に基づき、2次元共形場理論の真空を分離クエンチしたときのエンタングルメントエントロピー(EE)の振る舞いを調べた。連結区間のEEの振る舞いは、零質量自由Dirac場でも重力双対をもつ共形場理論でも準粒子描像によって理解できた。連結区間が分離をまたがない場合とまたぐ場合で、笠-高柳公式のconnectedな測地線の振る舞いが大きく変わる。とくに分離をまたぐ連結区間に対するconnectedな測地線は、AdS空間に伸びた分離の境界によって伸びて、これがlog増大の原因になることが分かった。AdS空間に伸びた分離の境界近くの重力側では非常に重い境界の存在によって起きた効果であると解釈できることが分かった。このとき測地線はPoincare座標で覆える領域を越えて伸びており、AdS空間の大域的な構造によって解釈される。これはAdS/BCFTの処方によって、上半平面上の2点関数であるEEを平面上の4点関数に帰着できたことで生じた効果である。

\ref{chap:doublequench}章では我々の論文\cite{Caputa:2019avh}に基づき、2次元共形場理論の真空を2重分離クエンチしたときのエンタングルメントエントロピー(EE)の振る舞いを調べた。クエンチの分離をまたがない連結領域のEEは零質量自由Dirac場でも重力双対を持つ共形場理論でも準粒子描像によって理解できた。このとき特に中央の領域$C$にある連結領域のEEの振る舞いは振動し、これは境界で準粒子が反射する描像として解釈できることが分かる。2つの境界をまたぐ連結区間のEEは零質量自由Dirac場でも重力双対を持つ共形場理論でも同様に、$t\to \infty$で、ブラックホールエントロピーの半分に対応した値に収束する。この振る舞いは、2つの分離によって中央の領域$C$は有限区間となったことで有限体積効果によって有限温度状態に熱化することで生じると考えられる。そして、2重分離クエンチにおける境界も1重分離クエンチと同様に非常に重い物体であると考えられるが、AdS$_3$の境界に残った重力相互作用の効果で引き寄せあうと考えられる。このことを2重分離クエンチと1重分離クエンチでのEEの振る舞いを比較することで確かめた。
\section*{今後の展望}
本研究では、2次元共形場理論の真空に対して分離クエンチ・2重分離クエンチをしたときのEEの振る舞いを調べた。

今後の課題として、別の共形場理論に対して分離クエンチを考えることは興味深い。
\begin{description}
	\item[2次元ユニタリーCFT] \hfill\\
	$c<1$の2次元CFTのユニタリーミニマルモデルは可積分な理論として知られており、典型的な熱化現象は見られないことが多くの例で知られている。その理論において2重分離クエンチをしたときにも、中央の領域$C$を含む大きな部分系でのEEを調べたときに、部分系の大きさを大きくする``熱力学極限"で熱化するような振る舞いが見られるかどうかを検証することは興味深い。また、$c>1$のLiouville理論などでのEEの振る舞いを調べることも、AdS/CFT対応やCFTのカオス性を調べる上で興味深い。
	\item[高次元共形場理論]
	分離クエンチ・2重分離クエンチは系の相互作用を切ることで系を分離するものであり、分離によって分かれた系は因果的に独立になる。$d+1\ (d\ge 2)$次元の共形場理論でも$d$次元の境界を作ることで、系の分離クエンチを考えることができる。このとき$d$次元の境界を``ホライズン"として分かれた系は因果的に独立になる。そのときのエンタングルメントのダイナミクスを笠-高柳公式などで調べることは、$d+2$次元の非平衡な重力理論を調べることにつながる。
\end{description}

また、本研究\cite{Shimaji:2018czt}\cite{Caputa:2019avh}では分離クエンチの他に、「接合クエンチ」も取り扱っている。接合クエンチとは、相互作用していなかった系を瞬時に相互作用させて接合するクエンチのことである。例えば有限区間と半直線を接合クエンチすること\cite{Calabrese_2016}は、直感的に言えば「熱平衡化した系」(=有限区間)を「低温の環境」(=半直線)にくっつけることを表しており、熱力学の問題として興味深い。このような考え方はブラックホール蒸発のトイモデルにも用いられ、\cite{almheiri2019page}では我々の論文\cite{Shimaji:2018czt}が引用されている。

\section*{謝辞}
修士課程に進む機会を与えてくださった島地屋餅店の皆様と、765プロダクション所属の横山奈緒さんに感謝いたします。

(理学部5号館5階の修論スペースに置かれるだろう論文には、研究室の方々や同期への感謝を表明していますが、公開verでは実名を出すことを避けておきます。)